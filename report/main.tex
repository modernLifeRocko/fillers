%!Tex program = lualatex
\documentclass{article}
\usepackage{geometry}
\usepackage{graphics}
\usepackage[backend=biber]{biblatex}



\title{Fillers in Anime}
\author{Andrés Fontalvo, PhD}

\begin{document}
\maketitle

\begin{abstract}
  Anyone who has watched anime for enough time has complained about the abundance of filler in some anime.
  We analyze their frequency in anime, how it is affected by length of series, and their evolution through time.
  We also speculate about the future of filler in anime and their effect on quality of the series.
\end{abstract}
  
\section{Filler Data}
\subsection{What is filler?}
Anime is usually an adaptation of material in other media.
Filler refers to episodes that don't follow the orignal material.
They do not move the plot forward.

While completely orignal anime do exist, they are the exception rather than the rule.



\subsection{Source of filler data}

\section{Exploratory Analysis}
\subsection{Shape of data: Mapper algorithm}
\subsection{Fillers in time}
  \begin{figure}[!h]
    \includegraphics{../images/plot_evolution.png}
    \caption{Percentage of filler vs Release Year}
  \end{figure}

\subsection{Fillers vs Quality}

\section{Conclusions}
\printbibliography
\end{document}
